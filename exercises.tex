% Notes and exercises from Set Theory by Jech
% By John Peloquin
\documentclass[letterpaper,12pt]{article}
\usepackage{amsmath,amssymb,amsthm,enumitem,fourier}

\newcommand{\N}{\boldsymbol{N}}
\newcommand{\Q}{\boldsymbol{Q}}
\newcommand{\I}{\boldsymbol{I}}
\newcommand{\R}{\boldsymbol{R}}
\newcommand{\C}{\boldsymbol{C}}
\renewcommand{\P}{P}
\newcommand{\Ord}{\mathit{Ord}}
\newcommand{\cc}{\mathfrak{c}}
\newcommand{\B}{\mathcal{B}}
\newcommand{\CC}{\mathcal{C}}
\newcommand{\Gd}{G_{\delta}}
\newcommand{\al}{\aleph}
\newcommand{\alo}{\al_{\omega}}
\newcommand{\aloo}{\al_{\omega_1}}

\newcommand{\union}{\cup}
\newcommand{\bigunion}{\bigcup}
\newcommand{\sect}{\cap}
\newcommand{\bigsect}{\bigcap}
\newcommand{\after}{\circ}
\newcommand{\mult}{\cdot}

\DeclareMathOperator{\dom}{dom}
\DeclareMathOperator{\ran}{ran}
\DeclareMathOperator{\type}{type}
\DeclareMathOperator{\cf}{cf}

\newcommand{\suc}[1]{#1\union\{#1\}}
\newcommand{\csuc}[1]{#1^+}
\newcommand{\preimage}[1]{#1_{-1}}
\newcommand{\card}[1]{|#1|}
\newcommand{\bigcard}[1]{\bigl|#1\bigr|}
\newcommand{\seq}[1]{\langle#1\rangle}
\newcommand{\cond}[1]{#1^*}
\newcommand{\abs}[1]{|#1|}

% Theorems
\theoremstyle{definition}
\newtheorem*{exer}{Exercise}

\theoremstyle{remark}
\newtheorem*{rmk}{Remark}

% Meta
\title{Notes and exercises from \textit{Set Theory}}
\author{John Peloquin}
\date{}

\begin{document}
\maketitle

\section*{Introduction}
This document contains notes and exercises from~\cite{jech}.

\section*{Chapter~1}
\begin{exer}[1.2]
There is no set~\(X\) such that \(\P(X)\subset X\).
\end{exer}
\begin{proof}
By the axiom of regularity~(1.8), \(X\)~is \(\in\)-minimal in~\(\{X\}\), so \(X\not\in X\) and hence \(\P(X)\not\subset X\).
\end{proof}

\begin{exer}[1.3]
If \(X\)~is inductive, then the set \(\{\,x\in X\mid x\subset X\,\}\) is inductive. Hence \(\N\)~is transitive and for each \(n\in\N\), \(n=\{\,m\in\N\mid m<n\,\}\).
\end{exer}
\begin{proof}
Let \(S=\{\,x\in X\mid x\subset X\,\}\). By inductivity of~\(X\), \(\emptyset\in S\), and if \(x\in S\), then \(\suc{x}\in S\), so \(S\)~is inductive. Taking \(X=\N\), it follows that \(S=\N\) since \(\N\)~is the smallest inductive set. Hence \(n\in\N\) implies \(n\subset\N\), so \(\N\)~is transitive and \(n=\{\,m\in\N\mid m<n\,\}\).
\end{proof}
\begin{rmk}
We proved transitivity of~\(\N\) ``by induction'' on~\(\N\): \(0\subset\N\) and if \(n\subset\N\) then \(n+1\subset\N\), so \(n\subset\N\) for all \(n\in\N\). The following exercises are similar.
\end{rmk}

\begin{exer}[1.4]
If \(X\)~is inductive, then the set \(\{\,x\in X\mid x\text{ is transitive}\,\}\) is inductive. Hence every \(n\in\N\) is transitive.
\end{exer}
\begin{proof}
The class~\(C\) of transitive sets is inductive. Indeed, \(\emptyset\)~is transitive, and if \(x\)~is transitive then \(\suc{x}\)~is transitive since \(y\in\suc{x}\) implies \(y\subset x\subset\suc{x}\). It follows that \(\{\,x\in X\mid x\text{ is transitive}\,\}=X\sect C\) is inductive since the intersection of two inductive classes is inductive. Taking \(X=\N\), it follows as above that every \(n\in\N\) is transitive.
\end{proof}

\begin{exer}[1.5]
If \(X\)~is inductive, then the set \(\{\,x\in X\mid x\text{ is transitive and }x\not\in x\,\}\) is inductive. Hence \(n\not\in n\) and \(n\ne n+1\) for all \(n\in\N\).
\end{exer}
\begin{proof}
The class \(C=\{\,x\mid x\text{ is transitive and }x\not\in x\,\}\) is inductive. Indeed, \(\emptyset\in C\). If \(x\in C\), then \(\suc{x}\)~is transitive (by inductivity of the class of transitive sets). Also \(\suc{x}\not\in x\), lest \(\suc{x}\subset x\) by transitivity of~\(x\) and hence \(x\in x\)---contradicting \(x\not\in x\). Similarly \(\suc{x}\not=x\). Therefore \(\suc{x}\not\in\suc{x}\). So \(\suc{x}\in C\), and \(C\)~is inductive. It follows as above that \(X\sect C\)~is inductive, and taking \(X=\N\) that \(n\not\in n\) and hence \(n\ne n+1\) for all \(n\in\N\).
\end{proof}

\begin{rmk}
In order to prove that \(n\not\in n\) for all \(n\in\N\) by induction on~\(\N\), we ``loaded the induction hypothesis'' with transitivity.
\end{rmk}

\begin{exer}[1.6]
If \(X\)~is inductive, then the set \(\{\,x\in X\mid x\text{ is transitive and regular}\,\}\) is inductive, where a set \(x\)~is called \emph{regular} if every nonempty subset of~\(x\) has an \(\in\)-minimal element.
\end{exer}
\begin{proof}
The class \(C=\{\,x\mid x\text{ is transitive and regular}\,\}\) is inductive. Indeed, \(\emptyset\in C\). If \(x\in C\), then \(\suc{x}\)~is transitive. If \(y\subset\suc{x}\) is nonempty, let \(z=y-\{x\}\subset x\). If \(z=\emptyset\), then \(y=\{x\}\), and \(x\)~is \(\in\)-minimal in~\(y\) by regularity of~\(x\). If \(z\ne\emptyset\) and \(t\)~is \(\in\)-minimal in~\(z\), then \(x\not\in t\) by transitivity and regularity of~\(x\), so \(t\)~is \(\in\)-minimal in~\(y\). Therefore \(\suc{x}\)~is regular. So \(\suc{x}\in C\), and \(C\)~is inductive. It follows as above that \(X\sect C\)~is inductive.
\end{proof}

\begin{rmk}
Taking \(X=\N\) it follows as above that every \(n\in\N\) is regular.
\end{rmk}

\begin{exer}[1.7]
Every nonempty \(X\subset\N\) has an \(\in\)-minimal element.
\end{exer}
\begin{proof}
Choose \(n\in X\). If \(n\)~is not \(\in\)-minimal in~\(X\), then \(n\sect X\)~is a nonempty subset of~\(n\), which has an \(\in\)-minimal element~\(m\) (Exercise~1.6). By transitivity of~\(n\) (Exercise~1.4), \(m\)~is \(\in\)-minimal in~\(X\).
\end{proof}

\begin{exer}[1.8]
If \(X\)~is inductive, then the set \(\{\,x\in X\mid x=\emptyset\text{ or }\exists y(x=\suc{y})\,\}\) is inductive. Hence for all \(n\in\N\), \(n=0\) or \(n=m+1\) for some \(m\in\N\).
\end{exer}
\begin{proof}
The class \(C=\{\,x\mid x=\emptyset\text{ or }\exists y(x=\suc{y})\,\}\) is obviously inductive, so \(X\sect C\)~is inductive. Taking \(X=\N\), the rest follows, with \(m\in\N\) by transitivity of~\(\N\) (Exercise~1.3).
\end{proof}

\begin{exer}[1.9]
Let \(A\subset\N\) be such that \(0\in A\), and if \(n\in A\) then \(n+1\in A\). Then \(A=\N\).
\end{exer}
\begin{proof}
\(A\)~is inductive, so \(A=\N\) since \(\N\)~is the smallest inductive set.

Alternately, if \(A\ne\N\), let \(m\)~be \(\in\)-minimal in \(\N-A\) (Exercise~1.7). Then \(m\ne0\), and \(m\ne n+1\) for any \(n\in\N\), which is impossible (Exercise~1.8).
\end{proof}

\begin{exer}[1.10]
Each \(n\in\N\) is T-finite.
\end{exer}
\begin{proof}
By induction on~\(\N\). First, \(n=0\)~is T-finite since \(\P(\emptyset)=\{\emptyset\}\) and \(\emptyset\)~is \(\subset\)-maximal in~\(\{\emptyset\}\). If \(n\)~is T-finite, suppose \(X\subset\P(\suc{n})\) is nonempty. Let
\[X'=\{\,u-\{n\}\mid u\in X\,\}\]
Then \(X'\subset\P(n)\) is nonempty and has a \(\subset\)-maximal element \(u-\{n\}\) for some \(u\in X\) by T-finiteness of~\(n\), where we may assume \(n\in u\) if \(u\union\{n\}\in X\). We claim \(u\)~is \(\subset\)-maximal in~\(X\). Indeed, if \(v\in X\) and \(u\subset v\), then \(u-\{n\}\subset v-\{n\}\), so \(u-\{n\}=v-\{n\}\) by \(\subset\)-maximality of~\(u-\{n\}\) in~\(X'\), so \(u=v\) by the assumption about~\(u\). Therefore \(n+1\)~is T-finite.
\end{proof}

\begin{exer}[1.11]
\(\N\)~is T-infinite. In fact, \(\N\subset\P(\N)\) has no \(\subset\)-maximal element.
\end{exer}
\begin{proof}
We know \(\N\subset\P(\N)\) by transitivity of~\(\N\) (Exercise~1.3), and for all \(n\in\N\) we have \(n\subset n+1\in\N\) but \(n\ne n+1\) (Exercise~1.5).
\end{proof}

\begin{rmk}
If \(A\)~is T-finite and \(\pi:A\to B\) is surjective (onto), then \(B\)~is T-finite.
\end{rmk}
\begin{proof}
By pullback. If \(X\subset\P(B)\) is nonempty, let
\[\preimage{X}=\preimage{\pi}(X)=\{\,\preimage{u}=\preimage{\pi}(u)\mid u\in X\,\}\]
Then \(\preimage{X}\subset\P(A)\) is nonempty and has a \(\subset\)-maximal element \(\preimage{u}\) for some \(u\in X\) by T-finiteness of~\(A\). We claim \(u\)~is \(\subset\)-maximal in~\(X\). Indeed, if \(v\in X\) and \(u\subset v\), then \(\preimage{u}\subset\preimage{v}\), so \(\preimage{u}=\preimage{v}\) by \(\subset\)-maximality of~\(\preimage{u}\) in~\(\preimage{X}\), so \(u=v\) by surjectivity of~\(\pi\). Therefore \(B\)~is T-finite.
\end{proof}

\begin{exer}[1.12]
Every finite set is T-finite.
\end{exer}
\begin{proof}
By definition, every finite set is the image of a T-finite natural number (Exercise~1.10), so is T-finite by the previous remark.

Alternately, if \(S\)~is T-infinite, choose \(X\subset\P(S)\) nonempty with no \(\subset\)-maximal element. By induction on~\(\N\), for each \(n\in\N\) there is a properly ascending chain \(u_0\subset\cdots\subset u_n\subset S\) of length~\(n+1\) in~\(X\), and hence an injection \(n\to S\) sending \(m\in n\) into \(u_{m+1}-u_m\). If \(S\)~had \(k\)~elements for some \(k\in\N\), there would then be an injection \(k+1\to k\), which is impossible by an easy induction on~\(k\). Therefore \(S\)~is infinite.
\end{proof}

\begin{exer}[1.13]
Every infinite set is T-infinite.
\end{exer}
\begin{proof}
If \(S\)~is infinite, let \(X\)~be the set of finite subsets of~\(S\). Clearly \(X\)~is nonempty since \(\emptyset\in X\). Also \(X\)~has no \(\subset\)-maximal element. Indeed, if \(u\in X\) then there is \(s\in S-u\) since \(S\)~is infinite, and \(u\union\{s\}\)~is finite (if \(u\)~has \(k\)~elements, then \(u\union\{s\}\)~has \(k+1\) elements), so \(u\subset u\union\{s\}\in X\) where the inclusion is proper. Therefore \(S\)~is T-infinite.
\end{proof}

\begin{rmk}
The previous two exercises show that (Cantor) finiteness is equivalent to Tarski finiteness in~ZF.
\end{rmk}

\section*{Chapter~2}
\begin{rmk}
In~(2.1), \(<\)~is actually a \emph{well-ordering} on~\(\Ord\). Indeed, if \(C\subset\Ord\) is a nonempty class of ordinals and \(\alpha\in C\) is not \(\in\)-minimal in~\(C\), then \(\alpha\sect C\)~is a nonempty subset of~\(\alpha\) which has an \(\in\)-minimal element~\(\beta\). By transitivity of~\(\alpha\), \(\beta\)~is \(\in\)-minimal in~\(C\).

This observation provides an alternative proof that \(\Ord\)~is a proper class: if \(\Ord\)~were a set, then because it is transitive and strictly well-ordered by~\(\in\), it would be an ordinal, and hence \(\Ord\in\Ord\)---contradicting strictness.
\end{rmk}

\begin{rmk}
In Definition~2.13, an ordinal is ``finite'' if and only if it is a ``finite ordinal''. In fact, if \(\alpha\)~is not a ``finite ordinal'' then \(\omega\subset\alpha\), and it follows by induction on~\(n\) that there is no surjection \(n\to\omega\) (every function \(n\to\omega\) is bounded), hence there is no surjection \(n\to\alpha\), so \(\alpha\)~is not ``finite''. The converse is trivial.
\end{rmk}

\begin{rmk}
In Theorem~2.27, the height of a well-ordering is just its order-type (ordinal), and the rank of an element in a well-ordering is just the order-type of the initial segment given by that element.
\end{rmk}
\begin{proof}
If \(P\)~is a well-ordering and \(P(x)=\{\,y\in P\mid y<x\,\}\), then
\[\type P=\sup_{x\in P}\{\,\type P(x)+1\,\}=\{\,\type P(x)\mid x\in P\,\}\]
Indeed, if \(x\in P\) then \(\type P(x)<\type P\) (Theorem~2.8), so \(\type P(x)+1\le\type P\). Conversely, if \(\alpha<\type P\), then \(\alpha=\type P(x)\) for some \(x\in P\), so \(\alpha<\type P(x)+1\). Taking \(P=P(x)\) yields
\[\type P(x)=\sup_{y<x}\{\,\type P(y)+1\,\}\]
The result now follows by uniqueness of rank.
\end{proof}

\begin{rmk}
If \(P\)~is a well-ordering and \(S\subset P\), then \(\type S\le\type P\).
\end{rmk}
\begin{proof}
By induction using the previous remark,
\[\type S=\sup_{x\in S}\{\,\type S(x)+1\,\}\le\sup_{x\in P}\{\,\type P(x)+1\,\}=\type P\qedhere\]
\end{proof}

\begin{exer}[2.2]
\(\alpha\)~is a limit ordinal if and only if \(\beta<\alpha\) implies \(\beta+1<\alpha\) for all~\(\beta\).
\end{exer}
\begin{proof}
If \(\alpha\)~is a limit ordinal and \(\beta<\alpha\), then \(\beta+1\le\alpha\) (2.5) and \(\beta+1\ne\alpha\), so \(\beta+1<\alpha\). If \(\alpha\)~is not a limit ordinal, then \(\beta+1=\alpha\) for some~\(\beta<\alpha\).
\end{proof}
\begin{rmk}
It follows that \(\alpha\)~is a nonzero limit ordinal if and only if it is inductive.
\end{rmk}

\begin{exer}[2.3]
If \(X\)~is inductive, then \(X\sect\Ord\)~is inductive. \(\N\)~is the least nonzero limit ordinal, where \(\N=\bigsect\{\,X\mid X\text{ inductive}\,\}\).
\end{exer}
\begin{proof}
Clearly \(\Ord\)~is inductive, so \(X\sect\Ord\)~is inductive since the intersection of two inductive classes is inductive. Taking \(X=\N\), it follows that \(\N\subset\Ord\), and since \(\N\)~is also transitive, \(\N\)~is an ordinal. By the previous remark, \(\N\)~is the least nonzero limit ordinal.
\end{proof}

\begin{exer}[2.4]
(Without the axiom of infinity.) Let \(\omega\)~be the least nonzero limit ordinal, if it exists, or~\(\Ord\) otherwise. The following are equivalent:
\begin{enumerate}[itemsep=0pt]
\item[(i)] There exists an inductive set.
\item[(ii)] There exists an infinite\footnote{In this exercise, a set is \emph{finite} if it is in bijective correspondence with some \(n\in\omega\) and \emph{infinite} otherwise, even if \(\omega=\Ord\).} set.
\item[(iii)] \(\omega\)~is a set.
\end{enumerate}
\end{exer}
\begin{proof}
(i) \(\iff\) (iii): The smallest inductive set is the least nonzero limit ordinal (Exercise~2.3).

(iii) \(\implies\) (ii): \(\omega\)~is infinite (Exercises 1.11--2).

(ii) \(\implies\) (i): For any finite set~\(A\), let \(\card{A}\)~denote the least \(n\in\omega\) in bijective correspondence with~\(A\) (the ``number of elements'' in~\(A\)). If \(X\)~is infinite, let
\[S=\{\,\card{A}\mid A\subset X\text{ finite}\,\}\]
Note \(S\)~is a set by replacement, and \(S\)~is inductive. Indeed, \(0\in S\) since \(\emptyset\subset X\) and \(\card{\emptyset}=0\). If \(n\in S\) and \(A\subset X\) with \(\card{A}=n\), then there must exist \(x\in X-A\) since \(X\)~is infinite, and \(\card{A\union\{x\}}=n+1\in S\).
\end{proof}
\begin{rmk}
This exercise shows that (i)--(iii) are equivalent forms of the axiom of infinity in~ZF.
\end{rmk}

\begin{exer}[2.5]
If \(W\)~is a well-ordered set, then there is no sequence \(\seq{\,a_n\mid n\in\N\,}\) in~\(W\) such that \(a_0>a_1>\cdots\)\ .
\end{exer}
\begin{proof}
If there were such a sequence, then \(\{\,a_n\mid n\in\N\,\}\) would be a nonempty subset of~\(W\) with no least element, contradicting well-ordering.
\end{proof}

\begin{exer}[2.6]
There are arbitrarily large limit ordinals.
\end{exer}
\begin{proof}
Given~\(\alpha\), let \(\beta=\alpha+\omega\). Clearly \(\beta>\alpha\). If \(\gamma<\beta\), then either \(\gamma<\alpha\), in which case \(\gamma+1<\alpha+1<\beta\), or \(\gamma\ge\alpha\), in which case \(\gamma=\alpha+n\) for some \(n\in\omega\) (Lemma~2.25), so \(\gamma+1=\alpha+n+1<\beta\). Thus \(\beta\)~is a limit ordinal (Exercise~2.2).
\end{proof}

\begin{rmk}[Chain rule]
If \(f,g:\Ord\to\Ord\) are nondecreasing and continuous, then so is~\(f\after g\). If \(f\)~and~\(g\) are normal, then so is~\(f\after g\).
\end{rmk}
\begin{proof}
If \(\alpha<\beta\), then \(g(\alpha)\le g(\beta)\), so \(f(g(\alpha))\le f(g(\beta))\). Let \(\alpha\)~be a nonzero limit ordinal. Clearly \(f(g(\alpha))\ge\lim_{\xi\to\alpha}f(g(\xi))\). If \(g(\alpha)=g(\xi')\) for some \(\xi'<\alpha\), then \(f(g(\alpha))=f(g(\xi'))\le\lim_{\xi\to\alpha}f(g(\xi))\). Otherwise, \(g(\alpha)\)~is a limit ordinal. Indeed, \(g(\alpha)=\lim_{\xi\to\alpha}g(\xi)\) by continuity of~\(g\), so if \(\beta<g(\alpha)\), then \(\beta<g(\xi)\) for some \(\xi<\alpha\), and hence \(\beta+1\le g(\xi)<g(\alpha)\). But then \(f(g(\alpha))=\lim_{\zeta\to g(\alpha)}f(\zeta)\) by continuity of~\(f\). If \(\zeta<g(\alpha)\), then \(\zeta<g(\xi)\) for some \(\xi<\alpha\), so \(f(\zeta)\le f(g(\xi))\), and hence \(\lim_{\zeta\to g(\alpha)}f(\zeta)\le\lim_{\xi\to\alpha}f(g(\xi))\). Therefore again \(f(g(\alpha))=\lim_{\xi\to\alpha}f(g(\xi))\) and \(f\after g\)~is continuous.

If \(f\)~and~\(g\) are also increasing, then \(f\after g\)~is increasing and hence normal.
\end{proof}

\begin{exer}[2.7]
Every normal sequence \(\seq{\,\gamma_{\alpha}\mid\alpha\in\Ord\,}\) has arbitrarily large fixed points (that is, \(\beta\)~such that \(\gamma_{\beta}=\beta\)).
\end{exer}
\begin{proof}
Given~\(\alpha\), let \(\beta_0=\gamma_{\alpha}\) and \(\beta_{n+1}=\gamma_{\beta_n}\) for all \(n\in\omega\). Note \(\beta_{n+1}\ge\beta_n\ge\alpha\) for all \(n\in\omega\) since \(\gamma\)~is increasing (Lemma~2.4). Let \(\beta=\lim_{n\to\omega}\beta_n\). Then
\[\gamma_{\beta}=\lim_{n\to\omega}\gamma_{\beta_n}=\lim_{n\to\omega}\beta_{n+1}=\beta\]
by the chain rule above (taking \(\beta_{\alpha}=\beta\) for all \(\alpha\ge\omega\)).
\end{proof}

\begin{exer}[2.8]
For all \(\alpha,\beta,\gamma\):
\begin{enumerate}[itemsep=0pt]
\item[(i)] \(\alpha\mult(\beta+\gamma)=\alpha\mult\beta+\alpha\mult\gamma\)
\item[(ii)] \(\alpha^{\beta+\gamma}=\alpha^{\beta}\mult\alpha^{\gamma}\)
\item[(iii)] \((\alpha^{\beta})^{\gamma}=\alpha^{\beta\mult\gamma}\)
\end{enumerate}
\end{exer}
\begin{proof}
By induction on~\(\gamma\).
\begin{enumerate}[itemsep=0pt]
\item[(i)] If \(\gamma=0\),
\[\alpha\mult(\beta+\gamma)=\alpha\mult(\beta+0)=\alpha\mult\beta=\alpha\mult\beta+0=\alpha\mult\beta+\alpha\mult 0=\alpha\mult\beta+\alpha\mult\gamma\]
If the result holds for~\(\gamma\), then
\begin{align*}
\alpha\mult(\beta+(\gamma+1))&=\alpha\mult((\beta+\gamma)+1)&&\text{by associativity of~\(+\)}\\
	&=\alpha\mult(\beta+\gamma)+\alpha&&\text{by definition of~\(\mult\)}\\
	&=(\alpha\mult\beta+\alpha\mult\gamma)+\alpha&&\text{by hypothesis}\\
	&=\alpha\mult\beta+(\alpha\mult\gamma+\alpha)&&\text{by associativity of~\(+\)}\\
	&=\alpha\mult\beta+\alpha\mult(\gamma+1)&&\text{by definition of~\(\mult\)}
\end{align*}
so the result holds for~\(\gamma+1\). If \(\gamma\)~is a nonzero limit ordinal and the result holds for all \(\xi<\gamma\), then
\begin{align*}
\alpha\mult(\beta+\gamma)&=\lim_{\xi\to\gamma}\alpha\mult(\beta+\xi)&&\text{by continuity of }\xi\mapsto\alpha\mult(\beta+\xi)\\
	&=\lim_{\xi\to\gamma}(\alpha\mult\beta+\alpha\mult\xi)&&\text{by hypothesis}\\
	&=\alpha\mult\beta+\alpha\mult\gamma&&\text{by continuity of }\xi\mapsto\alpha\mult\beta+\alpha\mult\xi
\end{align*}
so the result holds for~\(\gamma\). Note that continuity of the composite mappings involved follows from continuity of addition and multiplication and the chain rule above.
\item[(ii)] Similar.
\item[(iii)] Similar, using (i)~and~(ii) in the successor step.\qedhere
\end{enumerate}
\end{proof}

\begin{exer}[2.9]\
\begin{enumerate}[itemsep=0pt]
\item[(i)] \((\omega+1)\mult 2=\omega+1+\omega+1=\omega+\omega+1=\omega\mult 2+1<\omega\mult 2+2=\omega\mult 2+1\mult 2\)
\item[(ii)] \((\omega\mult 2)^2=\omega\mult 2\mult\omega\mult 2=\omega\mult\omega\mult 2=\omega^2\mult 2<\omega^2\mult 4=\omega^2\mult 2^2\)
\end{enumerate}
\end{exer}
\begin{rmk}
This result shows that \((\alpha+\beta)\mult\gamma\) does not in general equal \(\alpha\mult\gamma+\beta\mult\gamma\), and \((\alpha\mult\beta)^{\gamma}\) does not in general equal \(\alpha^{\gamma}\mult\beta^{\gamma}\).
\end{rmk}

\begin{exer}[2.10]
If \(\alpha<\beta\), then \(\alpha+\gamma\le\beta+\gamma\), \(\alpha\mult\gamma\le\beta\mult\gamma\), and \(\alpha^{\gamma}\le\beta^{\gamma}\) for all~\(\gamma\).
\end{exer}
\begin{proof}
By induction on~\(\gamma\).
\end{proof}

\begin{exer}[2.11]
\(2<3\) but
\begin{enumerate}[itemsep=0pt]
\item[(i)] \(2+\omega=\omega=3+\omega\)
\item[(ii)] \(2\mult\omega=\omega=3\mult\omega\)
\item[(iii)] \(2^{\omega}=\omega=3^{\omega}\)
\end{enumerate}
\end{exer}

\begin{exer}[2.12]
Let \(\epsilon_0=\lim_{n\to\omega}\alpha_n\) where \(\alpha_0=\omega\) and \(\alpha_{n+1}=\omega^{\alpha_n}\). Then \(\epsilon_0\)~is the least ordinal~\(\epsilon\) such that \(\omega^{\epsilon}=\epsilon\).
\end{exer}
\begin{proof}
By continuity of exponentiation and the chain rule above (taking \(\alpha_{\beta}=\epsilon_0\) for all \(\beta\ge\omega\)),
\[\omega^{\epsilon_0}=\lim_{n\to\omega}\omega^{\alpha_n}=\lim_{n\to\omega}\alpha_{n+1}=\epsilon_0\]
If \(\omega^{\epsilon}=\epsilon\), we prove by induction that \(\alpha_n\le\epsilon\) for all \(n\in\omega\), from which it follows that \(\epsilon_0\le\epsilon\). Indeed, since \(\omega^0=1\ne0\), we have \(\epsilon\ne 0\) and hence \(\alpha_0=\omega\le\omega^{\epsilon}=\epsilon\). If \(\alpha_n\le\epsilon\), then \(\alpha_{n+1}=\omega^{\alpha_n}\le\omega^{\epsilon}=\epsilon\).
\end{proof}

\begin{exer}[2.13]
A limit ordinal \(\gamma>0\) is indecomposable if and only if \(\alpha+\gamma=\gamma\) for all \(\alpha<\gamma\) if and only if \(\gamma=\omega^{\alpha}\) for some~\(\alpha>0\).
\end{exer}
\begin{proof}
If \(\gamma=\alpha+\beta\) with \(\alpha,\beta<\gamma\), then \(\alpha+\gamma>\alpha+\beta=\gamma\). Conversely, if \(\alpha<\gamma\) and \(\alpha+\gamma>\gamma\), fix~\(\beta\) such that \(\alpha+\beta=\gamma\) (Lemma~2.25). If \(\beta\ge\gamma\), then \(\gamma=\alpha+\beta\ge\alpha+\gamma>\gamma\), which is impossible, so \(\beta<\gamma\).

The forward direction of the second equivalence follows from the Cantor normal form (Theorem~2.26). For the reverse direction, we prove by induction on \(\alpha>0\) that \(\omega^{\alpha}\)~is indecomposable. The result holds for \(\alpha=1\) since \(n+\omega=\omega\) for all \(n\in\omega\). If \(\omega^{\alpha}\)~is indecomposable and \(\beta<\omega^{\alpha+1}=\omega^{\alpha}\mult\omega\), then \(\beta<\omega^{\alpha}\mult n\) for some \(n\in\omega\), so (Exercises 2.10 and 2.8)
\[\beta+\omega^{\alpha+1}\le\omega^{\alpha}\mult n+\omega^{\alpha}\mult\omega=\omega^{\alpha}\mult(n+\omega)=\omega^{\alpha}\mult\omega=\omega^{\alpha+1}\]
and hence \(\omega^{\alpha+1}\)~is indecomposable. Finally if \(\alpha>0\)~is a limit ordinal, \(\omega^{\xi}\)~is indecomposable for all \(\xi<\alpha\), and \(\beta<\omega^{\alpha}\), then by continuity
\[\beta+\omega^{\alpha}=\lim_{\xi\to\alpha}(\beta+\omega^{\xi})=\lim_{\xi\to\alpha}\omega^{\xi}=\omega^{\alpha}\]
and hence \(\omega^{\alpha}\)~is indecomposable.
\end{proof}

\section*{Chapter~3}
\begin{rmk}
If \(\kappa\) and~\(\lambda\) are ordinals which are cardinals, then \(\kappa\le\lambda\) in the \emph{ordinal} ordering if and only if \(\kappa\le\lambda\) in the \emph{cardinal} ordering~(3.2). Indeed, if \(\kappa\le\lambda\) in the ordinals, then \(\kappa\subset\lambda\), so \(\kappa\le\lambda\) in the cardinals. Conversely, if \(\kappa\le\lambda\) in the cardinals, then we cannot have \(\lambda<\kappa\) in the ordinals since \(\kappa\)~is a cardinal.
\end{rmk}

\begin{rmk}
An ordinal is a ``finite'' cardinal if and only if it is a ``finite cardinal''. In fact, if an ordinal is a ``finite'' cardinal, then it is a ``finite ordinal'' by the remark above, and hence it is a ``finite cardinal''. The converse is just the \emph{pigeonhole principle}, which is proved by induction on~\(\omega\).
\end{rmk}

\begin{rmk}
The arithmetic operations for finite cardinals in~(3.3) agree with the corresponding operations for finite ordinals.
\end{rmk}

\begin{exer}[3.1]\
\begin{enumerate}[itemsep=0pt]
\item[(i)] A subset of a finite set is finite.
\item[(ii)] A finite union of finite sets is finite.
\item[(iii)] The power set of a finite set is finite.
\item[(iv)] An image (projection) of a finite set is finite.
\end{enumerate}
\end{exer}
\begin{proof}\
\begin{enumerate}[itemsep=0pt]
\item[(i)] By an easy induction on \(n\in\N\), every subset of~\(n\) has \(m\)~elements for some \(m\in\N\) with \(m\le n\), from which the result follows.

Alternately, if \(B\subset A\), then \(\P(B)\subset\P(A)\), so \(X\subset\P(B)\) implies \(X\subset\P(A)\). If \(A\)~is finite, then \(A\)~is T-finite (Exercise~1.12), so \(B\)~is T-finite and hence \(B\)~is finite (Exercise~1.13).
\item[(ii)] If \(\card{A}=m\) and \(\card{B}=n\) and \(A\sect B=\emptyset\), then \(\card{A\union B}=m+n\). If \(A\sect B\ne\emptyset\), then \(A\union B\)~is a subset of the disjoint union,\footnote{Technically, \(A\union B\)~has the same cardinality as a subset of \((A\times\{0\})\union(B\times\{1\})\).} so \(\card{A\union B}\le m+n\) by~(i). Therefore the union of two finite sets is finite, and the union of any finite set of finite sets is finite by induction.
\item[(iii)] If \(\card{A}=n\), then \(\card{\P(A)}=2^n\) (Lemma~3.3).
\item[(iv)] If \(f:n\to B\) is surjective, define \(g:B\to n\) by letting \(g(b)\)~be the least \(m\in n\) such that \(f(m)=b\). Then \(g\)~is injective, so \(B\)~has the same cardinality as a subset of~\(n\), and hence \(\card{B}\le n\) by~(i). The result follows.\qedhere
\end{enumerate}
\end{proof}

\begin{exer}[3.2]\
\begin{enumerate}[itemsep=0pt]
\item[(i)] A subset of a countable set is at most countable.
\item[(ii)] A finite union of countable sets is countable.
\item[(iii)] An image (projection) of a countable set is at most countable.
\end{enumerate}
\end{exer}
\begin{proof}\
\begin{enumerate}[itemsep=0pt]
\item[(i)] If \(B\subset\N\) is infinite, let
\begin{align*}
b_0&=\text{least in }B\\
b_{n+1}&=\text{least in }B-\{b_0,\ldots,b_n\}\text{ (nonempty since \(B\)~is infinite)}
\end{align*}
Let \(C=\{\,b_n\mid n\in\N\,\}\), which is countable since \(n\mapsto b_n\) is a bijection. If \(B\ne C\), let \(b\)~be least in \(B-C\). There are only finitely many elements of~\(B\) less than~\(b\) (Exercise~3.1(i)), which must be \(b_0,\ldots,b_k\) for some \(k\in\N\) by hypothesis. But then \(b\)~is least in \(B-\{b_0,\ldots,b_k\}\), so \(b=b_{k+1}\in C\), which contradicts \(b\not\in C\). Therefore \(B=C\). The result follows.
\item[(ii)] Similar to the proof of Exercise~3.1(ii), replacing \(m\) and~\(n\) with~\(\al_0\) and using the fact that \(\al_0\le\card{A\union B}\le\al_0+\al_0=\al_0\) (Theorem 3.5).
\item[(iii)] Similar to the proof of Exercise~3.1(iv), replacing~\(n\) with~\(\N\).\qedhere
\end{enumerate}
\end{proof}

\begin{exer}[3.3]
\(\N\times\N\)~is countable.
\end{exer}
\begin{proof}
The mapping \((m,n)\mapsto 2^m(2n+1)-1\) is a bijection from \(\N\times\N\) to~\(\N\). In fact, it is injective by uniqueness of prime factorizations. If \(k\in\N\), let \(m\in\N\) be the highest power of~\(2\) dividing~\(k+1\). Then \(k+1=2^m(2n+1)\) for some \(n\in\N\), so \(k\)~is the image of~\((m,n)\).
\end{proof}

\begin{exer}[3.4]\
\begin{enumerate}[itemsep=0pt]
\item[(i)] The set of all finite sequences in~\(\N\) is countable.
\item[(ii)] The set of all finite subsets of a countable set is countable.
\end{enumerate}
\end{exer}
\begin{proof}\
\begin{enumerate}[itemsep=0pt]
\item[(i)] The set is at least countable since there are countably many sequences of length one, and it is at most countable since the mapping
\[(m_1,\ldots,m_n)\mapsto p_1^{m_1+1}\cdots p_n^{m_n+1}\]
is injective, where \(p_k\)~is the \(k\)-th prime.
\item[(ii)] The set of all finite subsets of~\(\N\) is at least countable since there are countably many singletons, and it is at most countable since it is the image of the mapping which takes each finite sequence in~\(\N\) to its underlying set (part~(i) and Exercise~3.2(iii)). The result follows.\qedhere
\end{enumerate}
\end{proof}

\begin{exer}[3.7]
If \(B\)~is a projection of~\(\omega_{\alpha}\), then \(\card{B}\le\al_{\alpha}\).
\end{exer}
\begin{proof}
If \(f:\omega_{\alpha}\to B\) is surjective, define \(g:B\to\omega_{\alpha}\) by letting \(g(b)\)~be the least \(\beta\in\omega_{\alpha}\) such that \(f(\beta)=b\). Then \(g\)~is injective, so \(\card{B}\le\al_{\alpha}\).
\end{proof}

\begin{exer}[3.9]
If \(B\)~is a projection of~\(A\), then \(\card{\P(B)}\le\card{\P(A)}\).
\end{exer}
\begin{proof}
If \(f:A\to B\) is surjective, define \(g:\P(B)\to\P(A)\) by \(g(X)=\preimage{f}(X)\). If \(g(X)=g(Y)\), then
\[X=f(g(X))=f(g(Y))=Y\]
by surjectivity of~\(f\), so \(g\)~is injective.
\end{proof}

\begin{exer}[3.10]
\(\omega_{\alpha+1}\)~is a projection of~\(\P(\omega_{\alpha})\).
\end{exer}
\begin{proof}
If \(X\subset\omega_{\alpha}\), then \(\type X\le\omega_{\alpha}\) by a remark about well-orderings above, so we can define \(f:\P(\omega_{\alpha})\to\omega_{\alpha+1}\) by \(f(X)=\type X\). If \(\beta\in\omega_{\alpha+1}\), then \(\beta\subset\omega_{\alpha}\) with \(f(\beta)=\beta\), so \(f\)~is surjective.
\end{proof}

\begin{exer}[3.11]
\(\al_{\alpha+1}<2^{2^{\al_{\alpha}}}\)
\end{exer}
\begin{proof}
\(\al_{\alpha+1}<2^{\al_{\alpha+1}}\le 2^{2^{\al_{\alpha}}}\), where the first inequality follows from Theorem~3.1 and the second inequality follows from Exercises~3.9--10.
\end{proof}

\begin{exer}[3.12]
If \(\al_{\alpha}\)~is an uncountable limit cardinal, then \(\cf\omega_{\alpha}=\cf\alpha\); also, \(\omega_{\alpha}\)~is the limit of a cofinal sequence \(\seq{\,\omega_{\alpha_{\xi}}\mid\xi<\cf\alpha\,}\) of cardinals.\footnote{The second part of the exercise in the book is incorrect. For example, \(\omega_{\omega+\omega}\)~is not the limit of a cofinal sequence \(\seq{\,\omega_n\mid n<\omega\,}\).}
\end{exer}
\begin{proof}
By Lemma~3.7(ii) and its proof, since \(\omega_{\alpha}=\lim_{\xi\to\alpha}\omega_{\xi}\).
\end{proof}

\begin{exer}[3.13]
(ZF) \(\omega_2\)~is not a countable union of countable sets.
\end{exer}
\begin{proof}
If \(\omega_2=\bigunion_{n<\omega} S_n\) where each~\(S_n\) is countable, let \(\alpha_n=\type S_n<\omega_1\). Then \(\alpha=\sup_n\alpha_n\le\omega_1\). For \((n,\xi)\in\omega\times\alpha\), let \(f(n,\xi)\)~be the \(\xi\)-th element of~\(S_n\) if \(\xi\in\alpha_n\) or the first element of~\(S_n\) if \(\xi\not\in\alpha_n\). Then \(f:\omega\times\alpha\to\omega_2\) is surjective. But \(0<\card{\omega\times\alpha}\le\al_0\mult\al_1=\al_1\), so \(\omega_2\)~is a projection of~\(\omega_1\) and \(\al_2\le\al_1\)---contradicting \(\al_1<\al_2\). Therefore \(\omega_2\)~is not a countable union of countable sets.
\end{proof}

\begin{exer}[3.14]
\(S\)~is D-infinite if and only if \(S\)~has a countable subset.
\end{exer}
\begin{proof}
If \(S\)~is D-infinite, let \(f:S\to X\) be a bijection with \(X\subset S\) and \(X\ne S\). Let \(s_0\in S-X\) and \(s_{n+1}=f(s_n)\) for all \(n\in\N\). By induction on~\(n\), \(s_n\ne s_m\) for all \(m<n\), so \(n\mapsto s_n\) is injective and \(\{\,s_n\mid n\in\N\,\}\) is a countable subset of~\(S\).

Conversely, if \(n\mapsto s_n\in S\) is injective, define \(f:S\to S\) by
\[f(s)=\begin{cases}
s_{n+1}&\text{if }s=s_n\\
s&\text{if }s\ne s_n\text{ for all }n
\end{cases}\]
Then \(f\)~is a bijection from~\(S\) to \(S-\{s_0\}\), so \(S\)~is D-infinite.
\end{proof}

\begin{rmk}
It follows that a D-finite subset of a countable set must be finite, since an infinite subset of a countable set is countable (Exercise~3.2(i)).
\end{rmk}

\begin{exer}[3.15]\
\begin{enumerate}[itemsep=0pt]
\item[(i)] If \(A\)~and~\(B\) are D-finite, then \(A\union B\) and~\(A\times B\) are D-finite.
\item[(ii)] The set of all finite injective sequences in a D-finite set is D-finite.
\item[(iii)] The union of a disjoint D-finite family of D-finite sets is D-finite.
\end{enumerate}
\end{exer}
\begin{proof}
In all cases we use the fact that a set is D-infinite if and only if it has a countable subset (Exercise~3.14).
\begin{enumerate}[itemsep=0pt]
\item[(i)] If \(A\union B\)~is not D-finite, let \(n\mapsto s_n\in A\union B\) be injective. If \(A\)~is D-finite, there must be~\(N\) such that for all \(n\ge N\), \(s_n\not\in A\), so \(s_n\in B\). But then \(n\mapsto s_{N+n}\in B\) is injective, so \(B\)~is not D-finite.

If \(A\times B\)~is not D-finite, let \(R\subset A\times B\) be countable. If \(\dom R\)~is infinite, then we can enumerate a countable subset of~\(A\) using an enumeration of~\(R\), so \(A\)~is not D-finite. If \(\dom R\)~is finite, then \(\ran R\)~must be infinite, so we can enumerate a countable subset of~\(B\) using an enumeration of~\(R\), and hence \(B\)~is not D-finite.
\item[(ii)] Let \(S\)~be a set. If \(n\mapsto f_n\) is injective with \(f_n:l_n\to S\) injective and \(l_n\in\N\) for all \(n\in\N\), choose \(k_0\)~least with \(l_{k_0}\ne 0\) and define
\begin{align*}
s_0&=f_{k_0}(0)&&\\
s_{n+1}&=f_k(m)&&k\text{ least such that }\ran f_k\not\subset\{s_0,\ldots,s_n\}\\
	&&&m\text{ least such that }f_k(m)\not\in\{s_0,\ldots,s_n\}
\end{align*}
Note \(k\)~must exist since there are only finitely many injective sequences in a finite set. Clearly \(n\mapsto s_n\) is injective, so \(S\)~is D-infinite.
\item[(iii)] Let \(S\)~be a set of disjoint D-finite sets. If \(n\mapsto s_n\in\bigunion S\) is injective, define
\begin{align*}
x_0&=\text{the unique set in~\(S\) containing~\(s_0\)}\\
x_{n+1}&=\text{the unique set in~\(S\) containing~\(s_k\)}\\
	&\quad k\text{ least such that }s_k\not\in\bigunion\{x_0,\ldots,x_n\}
\end{align*}
Note \(k\)~must exist since for each \(x\in S\) there is~\(N\) with \(s_n\not\in x\) for all \(n\ge N\). Clearly \(n\mapsto x_n\) is injective, so \(S\)~is D-infinite.\qedhere
\end{enumerate}
\end{proof}

\begin{exer}[3.16]
If \(A\)~is infinite, then \(\P(\P(A))\)~is D-infinite.
\end{exer}
\begin{proof}
The mapping \(n\mapsto\{\,X\subset A\mid\card{X}=n\,\}\) is injective (Exercise~3.14).
\end{proof}

\section*{Chapter~4}
\begin{rmk}
In Definition~4.2, a ``dense subset'' of a linear ordering is a subset that is ``dense'' (as a linear ordering itself), but the converse is not true. For example, the positive rationals form a subset of the reals that is ``dense'' (between any two positive rationals is another positive rational) but is not a ``dense subset'' (between two negative reals there is no positive rational).
\end{rmk}

\begin{rmk}
In Theorem~4.4, property~(iii) of a Dedekind cut is not needed for this purely order-theoretic construction.\footnote{See \cite{rudin}, Chapter~1, Exercise~20.}
\end{rmk}

\begin{exer}[4.1]
The set of all continuous functions \(f:\R\to\R\) has cardinality~\(\cc\), while \(\card{\R^{\R}}=2^{\cc}\).
\end{exer}
\begin{proof}
By definition \(\card{\R^{\R}}=\cc^{\cc}\), and \(2^{\cc}\le\cc^{\cc}\le(2^{\cc})^{\cc}=2^{\cc\mult\cc}=2^{\cc}\) since \(\cc\mult\cc=2^{\al_0}\mult 2^{\al_0}=2^{\al_0+\al_0}=2^{\al_0}=\cc\).

There are at least~\(\cc\) continuous functions since the constant functions are continuous. Conversely, there are at most~\(\cc\) since a continuous function on~\(\R\) is determined by its values on the dense subset~\(\Q\), so the mapping \(f\mapsto f\restriction\Q\) is injective, and \(\card{\R^{\Q}}=\cc^{\al_0}=(2^{\al_0})^{\al_0}=2^{\al_0\mult\al_0}=2^{\al_0}=\cc\).
\end{proof}

\begin{exer}[4.3]
The set of all algebraic reals is countable.
\end{exer}
\begin{proof}
The set is at least countable since the rationals are algebraic (\(m/n\)~is a root of the polynomial \(nx-m\)). Conversely, it is at most countable since there are only countably many nonzero polynomials with integer coefficients (Exercise~3.4(i)) each of which has only finitely many real roots, and a countable union of finite ordered sets is countable. (If \(S\)~is a countable family of finite ordered sets, then it is possible to enumerate a countable subset of~\(\bigunion S\) and to define an injection \(\bigunion S\to\omega\times\omega\).)
\end{proof}

\begin{exer}[4.4]
If \(S\subset\R\) is countable, then \(\card{\R-S}=\cc\).
\end{exer}
\begin{proof}
It is sufficient to prove that if \(S\subset\R\times\R\) is countable, then \(\card{\R\times\R-S}=\cc\), since \(\card{\R\times\R}=\card{\R}\). It cannot be that for every \(x\in\R\) there is \(y\in\R\) with \((x,y)\in S\), lest \(S\)~would be uncountable. Hence there is \(x\in\R\) with \(\{x\}\times\R\sect S=\emptyset\), that is, \(\{x\}\times\R\subset\R\times\R-S\), so \(\cc=\card{\{x\}\times\R}\le\card{\R\times\R-S}\le\cc\).
\end{proof}

\begin{exer}[4.5]\
\begin{enumerate}[itemsep=0pt]
\item[(i)] The set of all irrational numbers has cardinality~\(\cc\).
\item[(ii)] The set of all transcendental numbers has cardinality~\(\cc\).
\end{enumerate}
\end{exer}
\begin{proof}
By Exercises 4.3 and~4.4.
\end{proof}

\begin{exer}[4.6]
The set of all open sets of reals has cardinality~\(\cc\).
\end{exer}
\begin{proof}
The set has cardinality at least~\(\cc\) since \(r\mapsto(r,r+1)\) is injective. The set has cardinality at most~\(\cc\) since the mapping
\[O\mapsto\{\,(p,q)\in\Q\times\Q\mid\text{the interval }(p,q)\subset O\,\}\]
for open sets~\(O\) is injective by density of~\(\Q\) in~\(\R\), and \(\card{\P(\Q\times\Q)}=2^{\al_0}=\cc\).
\end{proof}

\begin{exer}[4.7]
The Cantor set~\(\C\) is perfect.
\end{exer}
\begin{proof}
\(\C\ne\emptyset\) since \(0\in\C\), \(\C\)~is closed since it is the complement in~\([0,1]\) of the union of the open intervals removed in the construction, and every point in~\(\C\) is the limit of a set of endpoints of those intervals, all of which are in~\(\C\).
\end{proof}

\begin{exer}[4.8]
If \(P\)~is perfect and \(P\sect(a,b)\ne\emptyset\), then \(\card{P\sect(a,b)}=\cc\).
\end{exer}
\begin{proof}
Define \(\alpha=\inf(P\sect(a,b))\) and \(\beta=\sup(P\sect(a,b))\). Then \(a\le\alpha<\beta\le b\) and \(\alpha,\beta\in P\). Moreover, \(P\sect[\alpha,\beta]\) is perfect, since in particular \(\alpha\) and~\(\beta\) are not isolated points. So \(\cc=\card{P\sect[\alpha,\beta]}\le\card{P\sect[a,b]}\) (Theorem~4.5), from which the result follows.
\end{proof}

\begin{exer}[4.9]
If \(F\)~is closed and \(P\not\subset F\) is perfect, then \(\card{P-F}=\cc\).
\end{exer}
\begin{proof}
If \(x\in P-F\), then by closure of~\(F\) there is~\((a,b)\) with \(x\in(a,b)\) and \(F\sect(a,b)=\emptyset\). Now \(P\sect(a,b)\ne\emptyset\) and \(P\sect(a,b)\subset P-F\), so \(\cc=\card{P\sect(a,b)}\le\card{P-F}\) (Exercise~4.8).
\end{proof}

\begin{exer}[4.10]
If \(P\)~is perfect, then \(\cond{P}=P\).
\end{exer}
\begin{proof}
If \(x\in\cond{P}\), then \(x\)~is a limit point of~\(P\), so \(x\in P\) by closure of~\(P\). Conversely if \(x\in P\sect(a,b)\), then \(P\sect(a,b)\)~is uncountable (Exercise~4.8), so \(x\in\cond{P}\).
\end{proof}

\begin{exer}[4.11]
If \(P\subset A\) is perfect, then \(P\subset\cond{A}\).
\end{exer}
\begin{proof}
\(P=\cond{P}\subset\cond{A}\) (Exercise~4.10).
\end{proof}

\begin{exer}[4.12]
If \(F\)~is an uncountable closed set and \(P\)~is the perfect subset of~\(F\) constructed in the Cantor-Bendixon theorem (Theorem~4.6), then \(P=\cond{F}\).
\end{exer}
\begin{proof}
We know \(P\subset\cond{F}\) (Exercise~4.11). If \(x\in\cond{F}\), then every neighborhood of~\(x\) contains uncountably many points of~\(F\), and hence a point of~\(P\) since \(F-P\)~is at most countable. Therefore \(x\in P\) by closure of~\(P\).
\end{proof}

\begin{exer}[4.13]
If \(F\)~is an uncountable closed set, then \(F=\cond{F}\union(F-\cond{F})\) is the unique partition\footnote{We allow that \(F-\cond{F}\)~may be empty.} of~\(F\) into a perfect subset and an at most countable subset.
\end{exer}
\begin{proof}
We know that \(\cond{F}\union(F-\cond{F})\) is such a partition (Exercise~4.12). If \(P\union S\)~is another such partition and \(P\ne\cond{F}\), then \(P\not\subset\cond{F}\) or \(\cond{F}\not\subset P\), so \(S\)~is uncountable or \(F-\cond{F}\)~is uncountable (Exercise~4.9)---a contradiction.
\end{proof}

\begin{exer}[4.14]
\(\Q\)~is not a countable intersection of open sets.
\end{exer}
\begin{proof}
If \(\Q=\bigsect O_n\) with \(O_n\)~open, then \(\Q\subset O_n\) and hence \(O_n\)~is dense in~\(\R\) for all~\(n\). It follows that \(\I=\bigunion(\R-O_n)\) is a countable union of closed nowhere dense sets, and since \(\Q\)~is a countable union of nowhere dense singletons, \(\R=\Q\union\I\) is a countable union of nowhere dense sets---contradicting the Baire category theorem (Theorem~4.8).
\end{proof}

\begin{exer}[4.15]
If \(B\)~is Borel and \(f:\R\to\R\) is continuous, then \(\preimage{f}(B)\)~is Borel.
\end{exer}
\begin{proof}
By induction on the Borel sets~\(\B\). Let
\[\CC=\{\,X\subset\R\mid\preimage{f}(X)\in\B\,\}\]
Then \(\R\in\CC\) since \(\preimage{f}(\R)=\R\in\B\). If \(X\in\CC\), then \(\preimage{f}(\R-X)=\R-\preimage{f}(X)\in\B\), so \(\R-X\in\CC\). If \(X_n\in\CC\), then \(\preimage{f}(\bigunion X_n)=\bigunion\preimage{f}(X_n)\in\B\), so \(\bigunion X_n\in\CC\). Finally, if \(O\)~is open, then \(\preimage{f}(O)\)~is open by continuity of~\(f\), so \(\preimage{f}(O)\in\B\) and \(O\in\CC\). Therefore \(\CC\)~is a \(\sigma\)-algebra containing the open sets, so \(\B\subset\CC\).
\end{proof}

\begin{exer}[4.16]
If \(f:\R\to\R\), the set of points where \(f\)~is continuous is~\(\Gd\).
\end{exer}
\begin{proof}
For \(\alpha>0\), \(f\)~is \emph{\(\alpha\)-continuous at~\(x\)} if there is \(\delta>0\) such that \(\abs{f(y)-f(z)}<\alpha\) for all \(y,z\in(x-\delta,x+\delta)\).\footnote{See \cite{abbott}, Definition~4.6.5.} If \(C\)~is the set of points where \(f\)~is continuous and \(C_n\)~is the set of points where \(f\)~is \(1/n\)-continuous, then \(C_n\)~is open for all~\(n\) and \(C=\bigsect C_n\).
\end{proof}

\section*{Chapter~5}
\begin{rmk}
If \(\kappa\)~is infinite and \(\card{A}\ge\kappa\), then \(\card{[A]^{<\kappa}}=\card{A}^{<\kappa}\).
\end{rmk}
\begin{proof}
Since \([A]^{<\kappa}=\bigunion_{\lambda<\kappa}[A]^{\lambda}\) is a partition, we have (Lemmas 5.7--8)
\[\card{[A]^{<\kappa}}=\sum_{\lambda<\kappa}\card{A}^{\lambda}=\kappa\mult\sup_{\lambda<\kappa}\card{A}^{\lambda}=\kappa\mult\card{A}^{<\kappa}=\card{A}^{<\kappa}\qedhere\]
\end{proof}

\begin{rmk}
An infinite cardinal~\(\kappa\) is singular if and only if \(\kappa=\sum_{i<\lambda}\kappa_i\) for cardinals \(\lambda<\kappa\) and \(\kappa_i<\kappa\) for all \(i<\lambda\).
\end{rmk}
\begin{proof}
If \(\kappa\)~is singular, let \(\lambda=\cf\kappa<\kappa\) and choose \(S_i\subset\kappa\) with \(\kappa_i=\card{S_i}<\kappa\) and \(\kappa=\bigunion_{i<\lambda}S_i\) (Lemma~3.10). Then
\[\kappa=\bigcard{\bigunion_{i<\lambda}S_i}\le\sum_{i<\lambda}\kappa_i\le\lambda\mult\kappa=\kappa\]
The converse follows from Lemma~3.10.
\end{proof}

\begin{rmk}
If \(\kappa_i\ge 2\) for \(1\le i\le n\), then \(\sum\kappa_i\le\prod\kappa_i\).
\end{rmk}
\begin{proof}
\[2^{n-1}\sum_{i=1}^n\kappa_i\le n\prod_{i=1}^n\kappa_i\le 2^{n-1}\prod_{i=1}^n\kappa_i\qedhere\]
\end{proof}

\begin{rmk}
Properties of singular cardinals may be determined by properties of regular cardinals below. For example, the continuum function is continuous at singular cardinals below which it is eventually constant (Corollary~5.17). This idea is extended by~SCH (Theorem~5.22).
\end{rmk}

\begin{exer}[5.2]
If \(X\)~is infinite and \(S\)~is the set of finite subsets of~\(X\), then \(\card{S}=\card{X}\).
\end{exer}
\begin{proof}
\(\card{X}\le\card{S}\) since \(x\mapsto\{x\}\) is injective. Conversely, \(S\)~is the projection of the set \(X^{<\omega}=\bigunion_{n<\omega}X^n\) of finite sequences in~\(X\) under the mapping which takes each sequence to its underlying set. Therefore (by~(5.2))
\[\card{S}\le\card{X^{<\omega}}=\sum_{n<\omega}\card{X}^n=\al_0\mult\card{X}=\card{X}\qedhere\]
\end{proof}
\begin{rmk}
This generalizes Exercise~3.4(ii).
\end{rmk}

\begin{exer}[5.3]
If \(P\)~is a linear ordering such that every initial segment of~\(P\) has cardinality \(<\kappa\), then \(\card{P}\le\kappa\).
\end{exer}
\begin{proof}
Call \(S\subset P\) \emph{cofinal} in~\(P\) if for all \(x\in P\) there is \(y\in S\) with \(x\le y\). If \(S\)~is not cofinal in~\(P\), choose \(F(S)\in P\) with \(F(S)>y\) for all \(y\in S\). By recursion on the ordinals, define
\[x_{\alpha}=F(\{\,x_{\xi}\mid\xi<\alpha\,\})\quad\text{if }\{\,x_{\xi}\mid\xi<\alpha\,\}\text{ is not cofinal in~\(P\)}\]
Let \(\theta\)~be the least ordinal at which the recursion cannot continue. If \(\theta=0\), then the result is trivial. If \(\theta=\alpha+1\), then \(x_{\alpha}\)~is greatest in~\(P\), so \(P=P(x_{\alpha})\union\{x_{\alpha}\}\) and \(\card{P}\le\kappa\). If \(\theta>0\) is a limit ordinal, then
\[P=\bigunion_{\alpha<\theta}P(x_{\alpha})\]
Also \(\al_0\le\card{\theta}\le\kappa\), lest \(\card{P(x_{\kappa+1})}\ge\kappa\). Therefore \(\card{P}\le\kappa\mult\kappa=\kappa\).
\end{proof}

\begin{exer}[5.4]
If \(A\)~can be well-ordered, then \(P(A)\)~can be linearly ordered.
\end{exer}
\begin{proof}
Identify \(P(A)\) with \(2^A\) and order \(2^A\) with the lexicographic order: \(f<g\) if \(f(x)<g(x)\) at the least \(x\in A\) where \(f(x)\ne g(x)\).
\end{proof}
\begin{rmk}
The lexicographic order is not in general a well-ordering. For example, consider \(2^{\N}\)~and the subset \(2^{\N}-\{0\}\).
\end{rmk}

\begin{exer}[5.5]
ZF~+~ZL implies~AC.
\end{exer}
\begin{proof}
If \(C\)~is a set of nonempty sets, let \(P\)~be the set of ``partial'' choice functions for~\(C\) partially ordered by inclusion: \(f:S\to\bigunion C\) with \(S\subset C\) and \(f(X)\in X\) for all \(X\in S\). Then \(\emptyset\in P\) and every chain in~\(P\) has its union as an upper bound. By ZL, there is a maximal element \(F\in P\). Now \(\dom F=X\), lest \(F\)~can be properly extended and is not maximal, so \(F\)~is a choice function for~\(C\).
\end{proof}

\begin{exer}[5.6]
ZF~+~CC implies that every infinite set has a countable subset.
\end{exer}
\begin{proof}
If \(X\)~is infinite, choose an injection \(f_n:n\to X\) for all \(n\in\N\) by~CC and recursively define \(g:\N\to X\) by \(g(n)=f_{2^n}(k)\) where \(k\)~is least such that \(f_{2^n}(k)\ne g(m)\) for all \(m<n\). Then \(g\)~is injective.
\end{proof}

\begin{exer}[5.7]
ZF~+~DC implies~CC.
\end{exer}
\begin{proof}
If \(C\)~is a countable set of nonempty sets, let \(n\mapsto C_n\in C\) be a bijection and assume without loss of generality that the~\(C_n\) are pairwise disjoint. (If they are not disjoint, replace them with~\(C_n\times\{n\}\).)

Define a relation~\(E\) on \(A=\bigunion C\) by
\[b\mathrel{E}a\iff(\exists n\in\N)(b\in C_{n+1}\land a\in C_n)\]
Note \(A\)~is nonempty and for every \(a\in A\) there is \(b\in A\) with \(b\mathrel{E}a\) since the~\(C_n\) are nonempty. By~DC, there is a sequence \(a_0,a_1,\ldots\in A\) with \(a_{n+1}\mathrel{E}a_n\) for all \(n\in\N\). By definition of~\(E\) and disjointness of the~\(C_n\), there is \(m\in\N\) such that \(a_n\in C_{n+m}\) for all \(n\in\N\). For \(0\le n<m\), choose \(a_{n-m}\in C_n\).\footnote{This does not require an axiom of choice since it involves only finitely many choices.} Then \(C_n\mapsto a_{n-m}\) (\(n\in\N\)) is a choice function for~\(C\).
\end{proof}

\begin{exer}[5.11]
\(\prod_{0<n<\omega}n=2^{\al_0}\).
\end{exer}
\begin{proof}
By Lemmas 5.9 and~5.6,
\[\prod_{0<n<\omega}n=(\sup_{0<n<\omega}n\ )^{\al_0}=\al_0^{\al_0}=2^{\al_0}\qedhere\]
\end{proof}

\begin{exer}[5.13]
\(\prod_{\alpha<\omega+\omega}\al_{\alpha}=\al_{\omega+\omega}^{\al_0}\)
\end{exer}
\begin{proof}
By (5.16) and Lemma~5.9,
\[\prod_{\alpha<\omega+\omega}\al_{\alpha}=\prod_{n<\omega}\al_n\mult\prod_{n<\omega}\al_{\omega+n}=\alo^{\al_0}\mult\al_{\omega+\omega}^{\al_0}=\al_{\omega+\omega}^{\al_0}\qedhere\]
\end{proof}

\begin{exer}[5.14]
If GCH holds, then (i) \(2^{<\kappa}=\kappa\) for infinite~\(\kappa\) and (ii) \(\kappa^{<\kappa}=\kappa\) for regular~\(\kappa\).
\end{exer}
\begin{proof}
For~(i),
\[2^{<\kappa}=\sup\{\,2^\mu\mid\mu<\kappa\,\}=\sup\{\,\csuc{\mu}\mid\mu<\kappa\,\}=\kappa\]
For~(ii), by Theorem~5.15,
\[\kappa^{<\kappa}=\sup\{\,\kappa^\mu\mid\mu<\kappa\,\}=\sup\{\,\kappa\mid\mu<\kappa\,\}=\kappa\qedhere\]
\end{proof}

\begin{exer}[5.17]
If \(\kappa\)~is infinite and \(0<\lambda<\cf\kappa\), then \(\kappa^{\lambda}=\sum_{\alpha<\kappa}\card{\alpha}^{\lambda}\).
\end{exer}
\begin{proof}
Since \(\kappa^{\lambda}=\bigunion_{\alpha<\kappa}\alpha^{\lambda}\) (Lemma~3.9(ii)),
\[\kappa^{\lambda}=\bigcard{\bigunion_{\alpha<\kappa}\alpha^{\lambda}}\le\sum_{\alpha<\kappa}\card{\alpha}^{\lambda}\le\kappa\mult\kappa^{\lambda}=\kappa^{\lambda}\qedhere\]
\end{proof}

\begin{exer}[5.18]
\(\alo^{\al_1}=\alo^{\al_0}\mult 2^{\al_1}\).
\end{exer}
\begin{proof}
By Exercise~5.19, taking \(\alpha=\omega\).
\end{proof}

\begin{exer}[5.19]
If \(\alpha<\omega_1\), then \(\al_{\alpha}^{\al_1}=\al_{\alpha}^{\al_0}\mult 2^{\al_1}\).
\end{exer}
\begin{proof}
By induction on~\(\alpha\). If \(\alpha=0\), then \(\al_{\alpha}^{\al_1}=2^{\al_1}=\al_{\alpha}^{\al_0}\mult 2^{\al_1}\) (Lemma~5.6). If the result holds for~\(\alpha\), then by the Hausdorff formula~(5.22),
\[\al_{\alpha+1}^{\al_1}=\al_{\alpha}^{\al_1}\mult\al_{\alpha+1}=\al_{\alpha}^{\al_0}\mult\al_{\alpha+1}\mult 2^{\al_1}=\al_{\alpha+1}^{\al_0}\mult 2^{\al_1}\]
so the result holds for~\(\alpha+1\). If \(\alpha>0\) is a limit ordinal and the result holds for all \(\xi<\alpha\), then \(\cf\al_{\alpha}=\cf\alpha=\al_0\) and (Lemma~5.19)
\[\al_{\alpha}^{\al_1}=\bigl(\lim_{\xi\to\alpha}\al_{\xi}^{\al_1}\bigr)^{\al_0}=\bigl(\lim_{\xi\to\alpha}(\al_{\xi}^{\al_0}\mult 2^{\al_1})\bigr)^{\al_0}\le\bigl(\al_{\alpha}^{\al_0}\mult 2^{\al_1}\bigr)^{\al_0}=\al_{\alpha}^{\al_0}\mult 2^{\al_1}\le\al_{\alpha}^{\al_1}\]
so the result holds for~\(\alpha\).
\end{proof}

\begin{exer}[5.24]
If \(2^{\al_0}\ge\alo\), then \(\alo^{\al_0}=2^{\al_0}\).
\end{exer}
\begin{proof}
By Theorem~5.20(ii).
\end{proof}
\begin{rmk}
\(2^{\al_0}\ne\alo\) by K\"onig's theorem (Corollary~5.12).
\end{rmk}

\begin{exer}[5.25]
If \(2^{\al_1}=\al_2\) and \(\alo^{\al_0}\ge\aloo\), then \(\aloo^{\al_1}=\alo^{\al_0}\).
\end{exer}
\begin{proof}
By Exercise~5.18,
\[\alo^{\al_0}\le\aloo^{\al_1}\le(\alo^{\al_0})^{\al_1}=\alo^{\al_1}=\alo^{\al_0}\mult 2^{\al_1}=\alo^{\al_0}\mult\aleph_2=\alo^{\al_0}\qedhere\]
\end{proof}

\begin{exer}[5.26]
If \(2^{\al_0}\ge\aloo\), then \(\gimel(\alo)=2^{\al_0}\) and \(\gimel(\aloo)=2^{\al_1}\).
\end{exer}
\begin{proof}
By Theorem~5.20(ii).
\end{proof}

\begin{exer}[5.27]
If \(2^{\al_1}=\al_2\), then \(\alo^{\al_0}\ne\aloo\).
\end{exer}
\begin{proof}
If \(\alo^{\al_0}=\aloo\), then \(\aloo^{\al_1}=\aloo\) (Exercise~5.25), which contradicts K\"onig's theorem (Corollary~5.14).
\end{proof}

% References
\begin{thebibliography}{0}
\bibitem{abbott} Abbott, Stephen. \textit{Understanding Analysis}. Springer, 2001.
\bibitem{jech} Jech, Thomas. \textit{Set Theory}, 3rd~ed. Springer, 2002.
\bibitem{rudin} Rudin, Walter. \textit{Principles of Mathematical Analysis}, 3rd~ed. McGraw-Hill, 1976.
\end{thebibliography}
\end{document}
